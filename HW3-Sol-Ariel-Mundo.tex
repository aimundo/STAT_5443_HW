% Options for packages loaded elsewhere
\PassOptionsToPackage{unicode}{hyperref}
\PassOptionsToPackage{hyphens}{url}
%
\documentclass[
]{article}
\usepackage{lmodern}
\usepackage{amssymb,amsmath}
\usepackage{ifxetex,ifluatex}
\ifnum 0\ifxetex 1\fi\ifluatex 1\fi=0 % if pdftex
  \usepackage[T1]{fontenc}
  \usepackage[utf8]{inputenc}
  \usepackage{textcomp} % provide euro and other symbols
\else % if luatex or xetex
  \usepackage{unicode-math}
  \defaultfontfeatures{Scale=MatchLowercase}
  \defaultfontfeatures[\rmfamily]{Ligatures=TeX,Scale=1}
\fi
% Use upquote if available, for straight quotes in verbatim environments
\IfFileExists{upquote.sty}{\usepackage{upquote}}{}
\IfFileExists{microtype.sty}{% use microtype if available
  \usepackage[]{microtype}
  \UseMicrotypeSet[protrusion]{basicmath} % disable protrusion for tt fonts
}{}
\makeatletter
\@ifundefined{KOMAClassName}{% if non-KOMA class
  \IfFileExists{parskip.sty}{%
    \usepackage{parskip}
  }{% else
    \setlength{\parindent}{0pt}
    \setlength{\parskip}{6pt plus 2pt minus 1pt}}
}{% if KOMA class
  \KOMAoptions{parskip=half}}
\makeatother
\usepackage{xcolor}
\IfFileExists{xurl.sty}{\usepackage{xurl}}{} % add URL line breaks if available
\IfFileExists{bookmark.sty}{\usepackage{bookmark}}{\usepackage{hyperref}}
\hypersetup{
  pdftitle={STAT 5443, HW3},
  pdfauthor={Ariel Mundo},
  hidelinks,
  pdfcreator={LaTeX via pandoc}}
\urlstyle{same} % disable monospaced font for URLs
\usepackage[margin=1in]{geometry}
\usepackage{color}
\usepackage{fancyvrb}
\newcommand{\VerbBar}{|}
\newcommand{\VERB}{\Verb[commandchars=\\\{\}]}
\DefineVerbatimEnvironment{Highlighting}{Verbatim}{commandchars=\\\{\}}
% Add ',fontsize=\small' for more characters per line
\usepackage{framed}
\definecolor{shadecolor}{RGB}{248,248,248}
\newenvironment{Shaded}{\begin{snugshade}}{\end{snugshade}}
\newcommand{\AlertTok}[1]{\textcolor[rgb]{0.94,0.16,0.16}{#1}}
\newcommand{\AnnotationTok}[1]{\textcolor[rgb]{0.56,0.35,0.01}{\textbf{\textit{#1}}}}
\newcommand{\AttributeTok}[1]{\textcolor[rgb]{0.77,0.63,0.00}{#1}}
\newcommand{\BaseNTok}[1]{\textcolor[rgb]{0.00,0.00,0.81}{#1}}
\newcommand{\BuiltInTok}[1]{#1}
\newcommand{\CharTok}[1]{\textcolor[rgb]{0.31,0.60,0.02}{#1}}
\newcommand{\CommentTok}[1]{\textcolor[rgb]{0.56,0.35,0.01}{\textit{#1}}}
\newcommand{\CommentVarTok}[1]{\textcolor[rgb]{0.56,0.35,0.01}{\textbf{\textit{#1}}}}
\newcommand{\ConstantTok}[1]{\textcolor[rgb]{0.00,0.00,0.00}{#1}}
\newcommand{\ControlFlowTok}[1]{\textcolor[rgb]{0.13,0.29,0.53}{\textbf{#1}}}
\newcommand{\DataTypeTok}[1]{\textcolor[rgb]{0.13,0.29,0.53}{#1}}
\newcommand{\DecValTok}[1]{\textcolor[rgb]{0.00,0.00,0.81}{#1}}
\newcommand{\DocumentationTok}[1]{\textcolor[rgb]{0.56,0.35,0.01}{\textbf{\textit{#1}}}}
\newcommand{\ErrorTok}[1]{\textcolor[rgb]{0.64,0.00,0.00}{\textbf{#1}}}
\newcommand{\ExtensionTok}[1]{#1}
\newcommand{\FloatTok}[1]{\textcolor[rgb]{0.00,0.00,0.81}{#1}}
\newcommand{\FunctionTok}[1]{\textcolor[rgb]{0.00,0.00,0.00}{#1}}
\newcommand{\ImportTok}[1]{#1}
\newcommand{\InformationTok}[1]{\textcolor[rgb]{0.56,0.35,0.01}{\textbf{\textit{#1}}}}
\newcommand{\KeywordTok}[1]{\textcolor[rgb]{0.13,0.29,0.53}{\textbf{#1}}}
\newcommand{\NormalTok}[1]{#1}
\newcommand{\OperatorTok}[1]{\textcolor[rgb]{0.81,0.36,0.00}{\textbf{#1}}}
\newcommand{\OtherTok}[1]{\textcolor[rgb]{0.56,0.35,0.01}{#1}}
\newcommand{\PreprocessorTok}[1]{\textcolor[rgb]{0.56,0.35,0.01}{\textit{#1}}}
\newcommand{\RegionMarkerTok}[1]{#1}
\newcommand{\SpecialCharTok}[1]{\textcolor[rgb]{0.00,0.00,0.00}{#1}}
\newcommand{\SpecialStringTok}[1]{\textcolor[rgb]{0.31,0.60,0.02}{#1}}
\newcommand{\StringTok}[1]{\textcolor[rgb]{0.31,0.60,0.02}{#1}}
\newcommand{\VariableTok}[1]{\textcolor[rgb]{0.00,0.00,0.00}{#1}}
\newcommand{\VerbatimStringTok}[1]{\textcolor[rgb]{0.31,0.60,0.02}{#1}}
\newcommand{\WarningTok}[1]{\textcolor[rgb]{0.56,0.35,0.01}{\textbf{\textit{#1}}}}
\usepackage{graphicx,grffile}
\makeatletter
\def\maxwidth{\ifdim\Gin@nat@width>\linewidth\linewidth\else\Gin@nat@width\fi}
\def\maxheight{\ifdim\Gin@nat@height>\textheight\textheight\else\Gin@nat@height\fi}
\makeatother
% Scale images if necessary, so that they will not overflow the page
% margins by default, and it is still possible to overwrite the defaults
% using explicit options in \includegraphics[width, height, ...]{}
\setkeys{Gin}{width=\maxwidth,height=\maxheight,keepaspectratio}
% Set default figure placement to htbp
\makeatletter
\def\fps@figure{htbp}
\makeatother
\setlength{\emergencystretch}{3em} % prevent overfull lines
\providecommand{\tightlist}{%
  \setlength{\itemsep}{0pt}\setlength{\parskip}{0pt}}
\setcounter{secnumdepth}{-\maxdimen} % remove section numbering

\title{STAT 5443, HW3}
\author{Ariel Mundo}
\date{2020-10-23}

\begin{document}
\maketitle

\hypertarget{problem-1}{%
\subsection{PROBLEM 1}\label{problem-1}}

\hypertarget{gibbs-sampler}{%
\subsubsection{GIBBS SAMPLER}\label{gibbs-sampler}}

This section contains parts 1-4 from PROBLEM 1.

\begin{Shaded}
\begin{Highlighting}[]
\KeywordTok{library}\NormalTok{(plyr)}
\KeywordTok{rm}\NormalTok{(}\DataTypeTok{list =} \KeywordTok{ls}\NormalTok{())}
\KeywordTok{set.seed}\NormalTok{(}\DecValTok{123}\NormalTok{)}
\NormalTok{alpha=}\DecValTok{2}
\NormalTok{beta_o=}\FloatTok{6.4}
\NormalTok{n=}\DecValTok{74}
\NormalTok{s_}\DecValTok{1}\NormalTok{=}\DecValTok{16}
\NormalTok{niter=}\DecValTok{1000}
\NormalTok{theta_init=s_}\DecValTok{1}\OperatorTok{/}\NormalTok{n}
\NormalTok{s_init=s_}\DecValTok{1}\OperatorTok{/}\NormalTok{n}
\NormalTok{  theta=}\KeywordTok{rep}\NormalTok{(}\OtherTok{NA}\NormalTok{,niter)}
\NormalTok{  s=}\KeywordTok{rep}\NormalTok{(}\OtherTok{NA}\NormalTok{,niter)}
\NormalTok{  theta[}\DecValTok{1}\NormalTok{] =}\StringTok{ }\NormalTok{theta_init}
\NormalTok{  s[}\DecValTok{1}\NormalTok{]=s_init}
  \ControlFlowTok{for}\NormalTok{(i }\ControlFlowTok{in} \DecValTok{2}\OperatorTok{:}\NormalTok{niter)\{}
\NormalTok{    s[i] =}\StringTok{ }\KeywordTok{rbinom}\NormalTok{(}\DecValTok{1}\NormalTok{,}\DataTypeTok{size=}\NormalTok{n,}\DataTypeTok{prob=}\NormalTok{theta[i}\DecValTok{-1}\NormalTok{])}
\NormalTok{    theta[i] =}\StringTok{ }\KeywordTok{rbeta}\NormalTok{(}\DecValTok{1}\NormalTok{,alpha}\OperatorTok{+}\NormalTok{s[i],beta_o}\OperatorTok{+}\NormalTok{n}\OperatorTok{-}\NormalTok{s[i])}
\NormalTok{  \}}
\NormalTok{  pmf<-}\KeywordTok{count}\NormalTok{(s)}
\NormalTok{  pmf}\OperatorTok{$}\NormalTok{vals<-pmf}\OperatorTok{$}\NormalTok{freq}\OperatorTok{/}\NormalTok{niter}
  
\KeywordTok{plot}\NormalTok{(theta,}\DataTypeTok{type=}\StringTok{"l"}\NormalTok{, }\DataTypeTok{main=}\StringTok{'Traceplot for theta'}\NormalTok{,}\DataTypeTok{xlab=}\StringTok{'iterations'}\NormalTok{)}
\end{Highlighting}
\end{Shaded}

\includegraphics{HW3-Sol-Ariel-Mundo_files/figure-latex/GIBBS SAMPLER-1.pdf}

\begin{Shaded}
\begin{Highlighting}[]
\KeywordTok{plot}\NormalTok{(s,}\DataTypeTok{type=}\StringTok{"l"}\NormalTok{, }\DataTypeTok{main=}\StringTok{'Traceplot for s'}\NormalTok{,}\DataTypeTok{xlab=}\StringTok{'iterations'}\NormalTok{)}
\end{Highlighting}
\end{Shaded}

\includegraphics{HW3-Sol-Ariel-Mundo_files/figure-latex/GIBBS SAMPLER-2.pdf}

\begin{Shaded}
\begin{Highlighting}[]
\KeywordTok{plot}\NormalTok{(}\DataTypeTok{x=}\NormalTok{pmf}\OperatorTok{$}\NormalTok{x,}\DataTypeTok{y=}\NormalTok{pmf}\OperatorTok{$}\NormalTok{vals, }\DataTypeTok{main=}\StringTok{'approximate pmf'}\NormalTok{)}
\end{Highlighting}
\end{Shaded}

\includegraphics{HW3-Sol-Ariel-Mundo_files/figure-latex/GIBBS SAMPLER-3.pdf}

\begin{Shaded}
\begin{Highlighting}[]
\KeywordTok{hist}\NormalTok{(s,}\DataTypeTok{freq=}\OtherTok{FALSE}\NormalTok{, }\DataTypeTok{main=}\StringTok{'Histogram for s'}\NormalTok{)}
\end{Highlighting}
\end{Shaded}

\includegraphics{HW3-Sol-Ariel-Mundo_files/figure-latex/GIBBS SAMPLER-4.pdf}

\begin{Shaded}
\begin{Highlighting}[]
\KeywordTok{print}\NormalTok{(}\KeywordTok{paste}\NormalTok{(}\StringTok{'posterior median of theta:'}\NormalTok{,}\KeywordTok{median}\NormalTok{(theta), }\StringTok{'MLE estimate'}\NormalTok{,s_}\DecValTok{1}\OperatorTok{/}\NormalTok{n))}
\end{Highlighting}
\end{Shaded}

\begin{verbatim}
## [1] "posterior median of theta: 0.230158313795905 MLE estimate 0.216216216216216"
\end{verbatim}

\begin{blue}

\emph{The estimated posterior mean of} \(\theta\) \emph{is indeed to the
maximum likelihood estimate of s/n .}

\end{blue}

\url{http://www.stat.cmu.edu/~brian/463-663/week10/Chapter\%2004.pdf}
\url{https://bookdown.org/rdpeng/advstatcomp/gibbs-sampler.html}

\hypertarget{part-5-of-problem-1}{%
\subsubsection{PART 5 OF PROBLEM 1}\label{part-5-of-problem-1}}

This section measures the sensitivity of the posterior median to the
choice of initial values. For this part, the previous Gibbs sampler will
be transformed in a function, and the initial ratio \(s/n\) will be
randomized and the median \(s\) and \(\theta\) will be plotted for all
iterations.

\begin{Shaded}
\begin{Highlighting}[]
\NormalTok{binomial_beta_gibbs<-}\ControlFlowTok{function}\NormalTok{(theta_init,s_init,iterations)\{}
\NormalTok{alpha=}\DecValTok{2}
\NormalTok{beta_o=}\FloatTok{6.4}
\NormalTok{n=}\DecValTok{74}
\NormalTok{  theta_init=s_}\DecValTok{1}\OperatorTok{/}\NormalTok{n}
\NormalTok{s_init=s_}\DecValTok{1}\OperatorTok{/}\NormalTok{n}
\NormalTok{  theta=}\KeywordTok{rep}\NormalTok{(}\OtherTok{NA}\NormalTok{,iterations)}
\NormalTok{  s=}\KeywordTok{rep}\NormalTok{(}\OtherTok{NA}\NormalTok{,iterations)}
\NormalTok{  theta[}\DecValTok{1}\NormalTok{] =}\StringTok{ }\NormalTok{theta_init}
\NormalTok{  s[}\DecValTok{1}\NormalTok{]=s_init}
  \ControlFlowTok{for}\NormalTok{(i }\ControlFlowTok{in} \DecValTok{2}\OperatorTok{:}\NormalTok{iterations)\{}
\NormalTok{    s[i] =}\StringTok{ }\KeywordTok{rbinom}\NormalTok{(}\DecValTok{1}\NormalTok{,}\DataTypeTok{size=}\NormalTok{n,}\DataTypeTok{prob=}\NormalTok{theta[i}\DecValTok{-1}\NormalTok{]) }\CommentTok{#conditional for s}
\NormalTok{    theta[i] =}\StringTok{ }\KeywordTok{rbeta}\NormalTok{(}\DecValTok{1}\NormalTok{,alpha}\OperatorTok{+}\NormalTok{s[i],beta_o}\OperatorTok{+}\NormalTok{n}\OperatorTok{-}\NormalTok{s[i]) }\CommentTok{#conditional for theta}
\NormalTok{  \}}
\NormalTok{  pmf<-}\KeywordTok{count}\NormalTok{(s)}
\NormalTok{  pmf}\OperatorTok{$}\NormalTok{vals<-pmf}\OperatorTok{$}\NormalTok{freq}\OperatorTok{/}\NormalTok{niter}
 \KeywordTok{return}\NormalTok{(}\KeywordTok{cbind}\NormalTok{(}\KeywordTok{median}\NormalTok{(s),}\KeywordTok{median}\NormalTok{(theta)))}
\NormalTok{\}}
\CommentTok{#v<-seq(0,10,0.1)}
\NormalTok{v<-}\KeywordTok{runif}\NormalTok{(}\DecValTok{1000}\NormalTok{)}
\NormalTok{test<-}\KeywordTok{matrix}\NormalTok{(}\OtherTok{NA}\NormalTok{,}\DecValTok{2}\NormalTok{,}\KeywordTok{length}\NormalTok{(v))}
\ControlFlowTok{for}\NormalTok{ (j }\ControlFlowTok{in} \DecValTok{1}\OperatorTok{:}\KeywordTok{length}\NormalTok{(v))\{}
\NormalTok{test[,j]<-}\KeywordTok{binomial_beta_gibbs}\NormalTok{(v[j],v[j],}\DecValTok{1000}\NormalTok{)}
\NormalTok{\}}
\KeywordTok{plot}\NormalTok{(test[}\DecValTok{2}\NormalTok{,], }\DataTypeTok{type=}\StringTok{'l'}\NormalTok{, }\DataTypeTok{xlab=}\StringTok{'iteration'}\NormalTok{,}\DataTypeTok{main=}\StringTok{'median theta from variable initial values'}\NormalTok{) }\CommentTok{#plotting the median values of all iterations}
\end{Highlighting}
\end{Shaded}

\includegraphics{HW3-Sol-Ariel-Mundo_files/figure-latex/PROBLEM 1 PART 5-1.pdf}

\begin{Shaded}
\begin{Highlighting}[]
\KeywordTok{plot}\NormalTok{(test[}\DecValTok{1}\NormalTok{,], }\DataTypeTok{type=}\StringTok{'l'}\NormalTok{, }\DataTypeTok{xlab=}\StringTok{'iteration'}\NormalTok{,}\DataTypeTok{main=}\StringTok{'median s from variable initial values'}\NormalTok{) }\CommentTok{#plotting the median values of all iterations}
\end{Highlighting}
\end{Shaded}

\includegraphics{HW3-Sol-Ariel-Mundo_files/figure-latex/PROBLEM 1 PART 5-2.pdf}

\begin{center}\rule{0.5\linewidth}{0.5pt}\end{center}

\hypertarget{problem-2}{%
\subsubsection{PROBLEM 2}\label{problem-2}}

The pmf in this case is the result of the combination of each individual
distribution:

\({n\choose s} \theta^{s}(1-\theta)^{n-s} \frac{\Gamma(\alpha_0+\beta_0)}{\Gamma(\alpha_0)\Gamma(\beta_0)}\theta^{\alpha_0-1}(1-\theta)^{\beta_0-1}\left[\frac{\exp(-\lambda)\lambda^{n}}{n!})\right]\)

This can be re-written as:
\(\frac{n!}{s!(n-s!)}\theta^{s}(1-\theta)^{n-s} \frac{\Gamma(\alpha_0+\beta_0)}{\Gamma(\alpha_0)\Gamma(\beta_0)}\theta^{\alpha_0-1}(1-\theta)^{\beta_0-1}\left[\frac{\exp(-\lambda)\lambda^{n}}{n!})\right]\)

For
\(\theta \vert s,n \propto \theta^{\alpha_0+s-1}(1-\theta)^{n-s+\beta_0-1} \propto Beta(\alpha_0+s,n-s+\beta_0)\)

For
\(s \vert \theta,n \propto {n\choose s} \theta^{s}(1-\theta)^{n-s} \propto Binomial (n,\theta)\)

Therefore the two terms \(n!\) cancel out and the exponential can be
taken out as it is a constant.

Writing the conditional distribution of \(n\) by considering only the
terms that include it and omitting constant terms:
\(f(n\vert\theta,s) \propto \frac{\lambda^{n}(1-\theta)^{n-s}}{(n-s)!}\)

Which, considering a constant \(\exp (-\lambda(1-\theta))\) means that
\((n-s) \propto Poisson(\lambda(1-\theta))\)

\begin{Shaded}
\begin{Highlighting}[]
\NormalTok{n_}\DecValTok{1}\NormalTok{=}\DecValTok{74}
\NormalTok{binomial_beta_gibbs_poisson<-}\ControlFlowTok{function}\NormalTok{(n_init,theta_init,s_init,iterations)\{}
\NormalTok{alpha=}\DecValTok{2}
\NormalTok{beta_o=}\FloatTok{6.4}
\NormalTok{lambda=}\DecValTok{64}
\NormalTok{  theta_bgp=}\KeywordTok{rep}\NormalTok{(}\OtherTok{NA}\NormalTok{,iterations)}
\NormalTok{  s_bgp=}\KeywordTok{rep}\NormalTok{(}\OtherTok{NA}\NormalTok{,iterations)}
\NormalTok{  n_bgp=}\KeywordTok{rep}\NormalTok{(}\OtherTok{NA}\NormalTok{,iterations)}
\NormalTok{    theta_bgp[}\DecValTok{1}\NormalTok{] =}\StringTok{ }\NormalTok{theta_init}
\NormalTok{    s_bgp[}\DecValTok{1}\NormalTok{]=s_init}
\NormalTok{    n_bgp[}\DecValTok{1}\NormalTok{]=n_init}
  \ControlFlowTok{for}\NormalTok{(i }\ControlFlowTok{in} \DecValTok{2}\OperatorTok{:}\NormalTok{iterations)\{}
\NormalTok{    s_bgp[i] =}\StringTok{ }\KeywordTok{rbinom}\NormalTok{(}\DecValTok{1}\NormalTok{,}\DataTypeTok{size=}\NormalTok{n_bgp[i}\DecValTok{-1}\NormalTok{],}\DataTypeTok{prob=}\NormalTok{theta_bgp[i}\DecValTok{-1}\NormalTok{])}
\NormalTok{    theta_bgp[i] =}\StringTok{ }\KeywordTok{rbeta}\NormalTok{(}\DecValTok{1}\NormalTok{,alpha}\OperatorTok{+}\NormalTok{s_bgp[i],beta_o}\OperatorTok{+}\NormalTok{n_bgp[i}\DecValTok{-1}\NormalTok{]}\OperatorTok{-}\NormalTok{s_bgp[i])}
\NormalTok{    n_bgp[i]=s_bgp[i]}\OperatorTok{+}\KeywordTok{rpois}\NormalTok{(}\DecValTok{1}\NormalTok{,lambda}\OperatorTok{*}\NormalTok{(}\DecValTok{1}\OperatorTok{-}\NormalTok{theta_bgp[i])) }\CommentTok{#conditional for n}
\NormalTok{  \}}
  \CommentTok{#pmf<-count(s)}
  \CommentTok{#pmf$vals<-pmf$freq/niter}
 \KeywordTok{return}\NormalTok{(}\KeywordTok{cbind}\NormalTok{(s_bgp,theta_bgp,n_bgp))}
\NormalTok{\}}

\NormalTok{test2<-}\KeywordTok{binomial_beta_gibbs_poisson}\NormalTok{(}\DecValTok{10}\NormalTok{,s_}\DecValTok{1}\OperatorTok{/}\NormalTok{n_}\DecValTok{1}\NormalTok{,s_}\DecValTok{1}\OperatorTok{/}\NormalTok{n_}\DecValTok{1}\NormalTok{,}\DecValTok{1000}\NormalTok{)}
\KeywordTok{hist}\NormalTok{(test2[,}\DecValTok{1}\NormalTok{],}\DataTypeTok{freq=}\OtherTok{FALSE}\NormalTok{, }\DataTypeTok{main=}\StringTok{'Histogram of s'}\NormalTok{)}
\end{Highlighting}
\end{Shaded}

\includegraphics{HW3-Sol-Ariel-Mundo_files/figure-latex/PROBLEM 2-1.pdf}

\begin{Shaded}
\begin{Highlighting}[]
\KeywordTok{print}\NormalTok{(}\KeywordTok{paste}\NormalTok{(}\StringTok{'Median theta of Binomial-Beta-Poisson'}\NormalTok{,}\KeywordTok{median}\NormalTok{(test2[,}\DecValTok{2}\NormalTok{]),}\StringTok{'The posterior median is similar to the method of problem 1'}\NormalTok{))}
\end{Highlighting}
\end{Shaded}

\begin{verbatim}
## [1] "Median theta of Binomial-Beta-Poisson 0.227290381804581 The posterior median is similar to the method of problem 1"
\end{verbatim}

Regarding convergence, plotting the traceplot for \(s\) and \(\theta\):

\begin{Shaded}
\begin{Highlighting}[]
\KeywordTok{plot}\NormalTok{(test2[,}\DecValTok{1}\NormalTok{], }\DataTypeTok{type=}\StringTok{'l'}\NormalTok{,}\DataTypeTok{main=}\StringTok{'traceplot for s'}\NormalTok{, }\DataTypeTok{xlab=}\StringTok{'iterations'}\NormalTok{)}
\end{Highlighting}
\end{Shaded}

\includegraphics{HW3-Sol-Ariel-Mundo_files/figure-latex/PROBLEM 2, 3-1.pdf}

\begin{Shaded}
\begin{Highlighting}[]
\KeywordTok{plot}\NormalTok{(test2[,}\DecValTok{2}\NormalTok{], }\DataTypeTok{type=}\StringTok{'l'}\NormalTok{,}\DataTypeTok{main=}\StringTok{'traceplot for theta'}\NormalTok{, }\DataTypeTok{xlab=}\StringTok{'iterations'}\NormalTok{)}
\end{Highlighting}
\end{Shaded}

\includegraphics{HW3-Sol-Ariel-Mundo_files/figure-latex/PROBLEM 2, 3-2.pdf}

\end{document}
